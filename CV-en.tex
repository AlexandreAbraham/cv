%% start of file `jdoe_classic.tex'.
%% Copyright 2006 Xavier Danaux.
%
% This work may be distributed and/or modified under the
% conditions of the LaTeX Project Public License version 1.3c,
% available at http://www.latex-project.org/lppl/.


\documentclass[10pt]{moderncv}

% moderncv styles
%\moderncvstyle{casual}       % optional argument are 'nocolor' (black & white cv) and 'roman' (for roman fonts, instead of sans serif fonts)
\moderncvstyle{classic}       % idem

% character encoding
%\usepackage[utf8]{inputenc}   % replace by the encoding you are using
%
\usepackage[T1]{fontenc}
\usepackage{graphicx}

% personal data (the given example is exhaustive; just give what you want)
\firstname{Alexandre}
\familyname{ABRAHAM}
\title{Machine Learning PhD}
\address{22 bis, rue Saint Vincent\\78580 Maule, \textsc{France}}  % for classic style
\phone{+33 6 24 98 86 56}
\email{abraham.alexandre@gmail.com}
\extrainfo{\weblink{http://www.twinee.fr}}
\photo[64pt]{zoom.png} % also optional, and the optional argument is the height the picture must be resized to
\quote{Any intelligent fool can make things bigger, more complex, and more violent. It takes a touch of genius -- and a lot of courage -- to move in the opposite direction.}% also optional
\hypersetup{backref,
colorlinks=true}


%\renewcommand{\listsymbol}{{\fontencoding{U}\fontfamily{ding}\selectfont\tiny\symbol{'102}}} % define another symbol to be used in front of the list items

%\renewcommand{\firstnamefont}{\fontsize{20}{22}\sffamily\mdseries\upshape}

% slanted small caps (only with roman family; the sans serif font doesn't exists :-()
%\usepackage{slantsc}
%\DeclareFontFamily{T1}{myfont}{}
%\DeclareFontShape{T1}{myfont}{m}{scsl}{ <-> cork-lmssqbo8}{}
%\usefont{T1}{myfont}{m}{scsl}Testing the font

% command and color used in this document, independently from moderncv
%\definecolor{see}{rgb}{0.5,0.5,0.5}% for web links

\newcommand{\up}[1]{\ensuremath{^\textrm{\scriptsize#1}}}% for text subscripts


%----------------------------------------------------------------------------------
%            content
%----------------------------------------------------------------------------------
\begin{document}
\maketitle
\vspace{-0.5cm}
%\makequote
%\cventry{years}{degree/job title}{institution/employer}{localization}{optionnal: grade/...}{optional: comment/job description}
%\color{firstnamecolor}\rule{\textwidth}{.25ex}
\enlargethispage{2cm}

\color{black}
\enlargethispage{1cm}
\vspace{-0.4cm}
\section{Thesis}
\cvitem{Title}{\textbf{Learning functional brain atlases modeling inter-subject
                       variability}}
\cvitem{Supervisors}{Dimitris SAMARAS (Stony Brook University -- Centrale
    Paris)\\ Gael VAROQUAUX (INRIA)}
\cvitem{Abstract}{
Recent studies have shown that resting-state spontaneous brain activity unveils
intrinsic cerebral functioning and complete information brought by prototype task
study. From these signals, we will set up a functional atlas of the brain, along 
with an across-subject variability model. The novelty of our approach lies in the
integration of neuroscientific priors and inter-individual variability in a
probabilistic description of the rest activity. These models will be applied to
large datasets. This variability, ignored until now, may lead to learning of fuzzy
atlases, thus limited in term of resolution. This program yields both numerical and
algorithmic challenges because of the data volume but also because of the complexity
of modelisation.}

\cvitem{Research Interests}{
            Resting State Functional Magnetic Resonance Imaging\\
            Sparse dictionary learning (with TV+L1 regularization: [MICCAI
            2013])\\
            Stochastic Methods\\
            Clustering\\
        Parallel / Grid computing}


\closesection{}
\vspace{-0.4cm}

\section{Publications}
\cvitem{MICCAI 2013\\Young scientist Award}{\textbf{Extracting brain regions from rest fMRI with
    total-variation constrained dictionary learning} -- \textit{
        Alexandre ABRAHAM, Elvis DOHMATOB, Bertrand THIRION, Dimitris SAMARAS
and Gael VAROQUAUX}}

\closesection{}

\vspace{-0.4cm}

\section{Education}
\vspace{-0.6cm}
\cventry{December 2012 -- Today}{Machine Learning PhD
Student}{\includegraphics[width=2.5cm]{inria.jpg}}{Neurospin Saclay}{INRIA --
CEA -- Paris Sud University}{In the Parietal Team leaded by Bertrand THIRION
\textit{See above for details}}{}

\vspace{-0.5cm}
\cventry{2009 -- 2010}{Computer Sciences Master's Degree}{\includegraphics[width=2.4cm]{upmc.png}}{Paris}{Artificial Intelligence and Decision}{\small Machine Learning, Decision, Intelligent Agents, Fuzzy Logic, Multi Agent Systems\ldots}

\vspace{-0.7cm}
\cventry{2007 -- 2009}{Researcher Student}{\includegraphics[width=1.3cm]{lrde_big.png}}{Kremlin-Bic�tre}{EPITA Research and Development Laboratory}{\small Olena Project -- generic and efficient programming in C++ applied to image processing. Research activities, software engineering, ConceptC++ study, implementation of a watershed algorithm.}

\vspace{-0.4cm}
\cventry{2004 -- 2009}{Computer Science Engineering School}{\includegraphics[width=1.5cm]{epita_light.jpg}}{Kremlin-Bic�tre}{EPITA -- 2:1 engineer degree}{\small Multi Agent System Simulation Platform: ant colony. Specialization in research and artificial intelligence. Development of projects using neural networks, genetic algorithm, text mining, metaheuristic\ldots}

%\cventry{2003}{Baccalaur�at G�n�ral Scientifique sp� SVT}{Institution Notre Dame}{Sannois}{Mention bien}{}
\closesection{}
\vspace{-0.4cm}
\section{Professional Experiences}

\vspace{-0.6cm}
\cventry{April 2012 -- Novembere 2012}{Research and Development
engineer}{\includegraphics[width=2.5cm]{inserm.jpg}}{Neurospin Saclay}{Inserm --
CEA}{Development of NISL, an API for neuro-imaging using Scikit Learn}

\vspace{-0.4cm}
\cventry{January 2011 -- March 2012}{IT consultant}{\includegraphics[width=3cm]{harmonie_logo.png}}{Paris}{Cr�dit Agricole Technologies}{Java expert -- mobile application server}

\vspace{-0.4cm}

\cventry{April -- September 2010}{AI Internship}{\includegraphics[width=3cm]{thales.jpg}}{\'Elancourt}{Thales Air Systems -- Artificial Intelligence Laboratory}{\small Adding goals to a multi-agent system}

\vspace{-0.4cm}

\cventry{October 2009 -- January 2010}{LibCINI Packaging}{\includegraphics[width=1.6cm]{upmc.png}}{Paris}{C initiation library}{
Conception of a Windows installer and the architecture allowing to generate it.}

\vspace{-0.6cm}
\cventry{March 2009 -- August 2009}{Research engineer internship -- SMACH
Project}{\includegraphics[width=1.8cm]{edf.jpg}\hspace{.5cm}\includegraphics[width=.7cm]{lip6.png}}{Paris}{Human behaviour simulation, multi agent system written in Java}
{\small Implementation of the new data model, GUI refactoring, Reinforcement learning.}

\vspace{-0.6cm}
\cventry{September 2007 -- February 2008}{C++ programming internship}{\includegraphics[width=2.2cm]{AR.png}}{Paris}{\small Maintenance and development of the embedded software}
{}
\closesection{}


%\section{Master thesis}
%\cvitem{title}{\emph{On the design of modern curriculum vit\ae{}s}}
%\cvitem{supervisors}{Pr P. Picasso and Pr G. Klimt}
%\cvitem{description}{\small Study of the complex design of a curriculum vit\ae{}, also known as ``résumé''. In my opinion, a good design needs to be show the personality of its author. Some people will thus prefer a more classic style, and others will be more audacious\dots}

\section{Teaching}
\vspace{-0.2cm}
\cventry{2009 -- 2010}{Teacher for secondary education}{\includegraphics[width=3cm]{oiio.png}}{Paris}{Mathematics, computer architecture, algorithms}{}

\vspace{-0.4cm}
\cventry{2006 -- 2010}{Occasional teacher}{\includegraphics[width=1.5cm]{epita_light.jpg}}{Kremlin-Bic�tre}{Programming, mathematics, algorithms}{}
\closesection{}

\section{Skills}
\cvcomputer{Operating Systems}{\textbf{Linux, Windows,}
BSD}{Softwares}{Visual Studio, Netbeans, Photoshop, Office suite}
\cvcomputer{Programming}{\textbf{Python, OCaml, Java, C/C++/C\#,} Matlab,
Prolog}{Scripting}{PHP, Shell, Python}
\cvcomputer{Scientic}{Scilab, Matlab}{Typography}{\LaTeX}
\cvcomputer{Web design}{XHTML, CSS, Javascript}{Database}{MySQL, PostGRESQL}
\cvcomputer{Artificial Intelligence}{\textbf{MAS, Machine Learning,} Decision, Data Mining}{MAS Frameworks}{Jade, Jason}
\closesection{}

\section{Languages}
\cvlanguage{Fran�ais}{Mother tongue}{}
\cvlanguage{Anglais}{Fluent (TOEIC: 985)}{}
\closesection{}

\section{Interests}
\cvitem{Hobbies}{\small Science Fiction novels, board games, role playing games}
\closesection{}
%\section{Section with a list}
%\cvlistitem{Single item.}
%\cvlistitem{Another single item.}
%\cvlistdoubleitem{Double\dots{}}{\dots{} item.}
%\cvlistdoubleitem{Another double\dots{}}{\dots{} item.}
%\closesection{}

%\section{Section with your own content}\closesection
%Your content here, inside the normal \LaTeX{} environment. You can use any regular \LaTeX{} command, display mathematics
%\[e =m\,c^2,\]
%put some table or figure, \dots

%\emptysection{}
%\cvitem{Now}{Back to moderncv layout, without making a new section :-)}

%\nocite{*}
%\bibliographystyle{plain}
%\bibliography{jdoe_publications}

\end{document}


%% end of file `jdoe_classic.tex'.
